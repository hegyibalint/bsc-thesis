\chapter{Programmable Realtime Unit}
\label{chap:pru}

This chapter will discuss the capabilities, and usage of the Programmable Realtime Unit (\textls{PRU}).

\section{Overview}

Even with realtime operating systems, no process can allocate hundred percent time of a CPU. There are essential OS management tasks, which handles the hardware. These management tasks needs to be executed periodically.
\\[1ex]
Other problems arise when the software specification requires communication protocols with precision frequency needs e.g. \isc, or \uart protocol. Where no \isc or \uart hardware is available, and the protocol is implemented in software. This software implementation is called \emph{bit--banging}, because the software is toggling the \gpio pins according to protocol specification.

Let's review with the \isc bit--banging example, and what is the problem with it.

\subsection{Example}

Briefly about the \isc protocol:
\begin{itemize}
	\item Master--slave architecture, multiple slaves can attached to the same signal line.
	\item Two signal line
	\begin{itemize}
		\item Clock (SDC)
		\item Data (SDA)
	\end{itemize}
	\item Bus frequencies can be arbitrary
\end{itemize}

\noindent\begin{minipage}{\textwidth}
	Briefly about the \uart protocol:
	\begin{itemize}
		\item P2P architecture
		\item Two signal line
		\begin{itemize}
			\item Receive (RDX)
			\item Transmission (TRX)
		\end{itemize}
		\item Bus frequency needs to be precise ($\le\approx\pm3\%$)
	\end{itemize}
\end{minipage}

\begin{figure}
\centering
\resizebox{\textwidth}{!}{
	\begin{tikzpicture}[
		squarednode/.style={
			rectangle,
			draw=black,
			very thick,
			minimum size=2cm
		},
		connection/.style={
			thick,
			shorten >=2pt,
			decoration={
				markings,
				mark={
					at position 1
					with {
						\draw[radius=3pt,fill=black] circle;
					}
				}
			},
			postaction=decorate
		}
	]
		\node[squarednode] (master) {Master};
		\node[squarednode] (slave1) [right = of master] {Slave$_1$};
		\node[squarednode] (slave2) [right = of slave1] {Slave$_2$};
		\node[squarednode] (slave3) [right = of slave2] {Slave$_3$};

		\draw[connection] (master.north) ++(-0.25,0) -- ++(0,1);
		\draw[connection] (slave1.north) ++(-0.25,0) -- ++(0,1);
		\draw[connection] (slave2.north) ++(-0.25,0) -- ++(0,1);
		\draw[connection] (slave3.north) ++(-0.25,0) -- ++(0,1);

		\draw[connection] (master.north) ++(0.25,0) -- ++(0,2);
		\draw[connection] (slave1.north) ++(0.25,0) -- ++(0,2);
		\draw[connection] (slave2.north) ++(0.25,0) -- ++(0,2);
		\draw[connection] (slave3.north) ++(0.25,0) -- ++(0,2);

		\draw[thick] (master.north) ++(-1,2) -- ++(13,0) node[pos=.95, above] {SDC};
		\draw[thick] (master.north) ++(-1,1) -- ++(13,0) node[pos=.95, below] {SDA};
	\end{tikzpicture}
	}%
	\caption{An example \isc configuration}
	\label{fig:i2c_example}
\end{figure}

\begin{figure}
\centering
\resizebox{\textwidth}{!}{
	\begin{tikzpicture}[
		squarednode/.style={
			rectangle,
			draw=black,
			very thick,
			minimum height=2cm,
			minimum width=2.5cm
		},
		label/.style={
			font=\footnotesize
		}
	]
		\node[squarednode] (A) [] {UART$_A$};
		\node[squarednode] (B) [right = 3cm of A] {UART$_B$};

		\draw[thick,->] (A.east) ++(0,0.5) -- ++(3,0) node[label,pos=-0.15] {TX} node[label,pos=1.15] {RX};
		\draw[thick,->] (B.west) ++(0,-0.5) -- ++(-3,0) node[label,pos=-0.15] {TX} node[label,pos=1.15] { RX};
	\end{tikzpicture}
	}%
	\caption{An example \uart configuration}
	\label{fig:example_uart}
\end{figure}

\vspace{2.5ex}

The master is responsibile to generate the clock signal, which both sides will be using to send data. This is a good example to depict the weakness of realtime operating systems, because we need a precise clock under the transmission.
